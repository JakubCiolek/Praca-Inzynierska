\chapter{Źródła}


\begin{figure}
	\begin{lstlisting}
		private void OnTriggerEnter2D(Collider2D collider)
		{
			humanScript otherHuman = collider.GetComponent<humanScript>();
			int conversation = UnityEngine.Random.Range(0,100);
			if (conversation == 1)
			{
				otherHuman.Conversation();
				Conversation();
			}
			if(otherHuman.GetStatus() == Status.INFECTED && status==Status.HEALTHY)
			{
				ShowRange();
				timeEnter = Time.time;
			}
			if(status == Status.INFECTED && otherHuman.GetStatus() == Status.HEALTHY)
			{
				ShowRange();
			}
		}
	\end{lstlisting}
	\caption{Funkcja obsługująca początek kontaktu miedzy agentami.}
	\label{fig:kod:OnTriggerEnter}
\end{figure}


\begin{figure}
	\begin{lstlisting}
		private void OnTriggerExit2D(Collider2D collider)
		{
			humanScript otherHuman = collider.GetComponent<humanScript>();
			if(otherHuman.GetStatus() == Status.INFECTED && status==Status.HEALTHY)
			{
				timeExit = Time.time;
				
				float contactDuration = timeExit - timeEnter;
				
				if (contactDuration > timeToInfection)
				{
					int exposed = UnityEngine.Random.Range(0,100);
					if(exposed < virusSpreadFactor)
					{
						status = Status.EXPOSED;
						body.color = Color.yellow;
						simInterface.IncreaseExposed();
						CalculateInfection();
					}
				}
			}
			HideRange();
		}
	\end{lstlisting}
	\caption{Funkcja obsługująca koniec kontaktu miedzy agentami.}
	\label{fig:kod:OnTriggerExit}
\end{figure}


\begin{figure}
	\begin{lstlisting}
		privatevoid CalculateInfection()
		{
			int infectionProbabilty = (int)(virusSpreadFactor*(1 - immunity));
			int attempt = UnityEngine.Random.Range(1,100);
			if(attempt < infectionProbabilty)
			{
				infectionTime = clock.GetHoursPassed() + UnityEngine.Random.Range(incubationPeriod-24, incubationPeriod+24);
				willBeInfected = true;
			}
			else
			{
				infectionTime = clock.GetHoursPassed() + 24;
				willBeInfected = false;
			}
		}
	\end{lstlisting}
	\caption{Funkcja wyliczająca szansę na rozwinięcie się choroby.}
	\label{fig:kod:CalculateInfection}
\end{figure}

\begin{figure}
	\begin{lstlisting}
		private void StartActivity()
		{
			activity = (Activity)UnityEngine.Random.Range(0,4);
			bool seatFound = false;
			switch(activity)
			{
				case Activity.WORK:
				seatFound = SearchSeat(desks);
				break;
				case Activity.BREAK:
				seatFound = SearchSeat(sofas);
				break;
				case Activity.LUNCH:
				seatFound = SearchSeat(kitchenSeats);
				break;
				default:
				break;
			}
			if(!seatFound)
			{
				activity = Activity.RANDOM_WALK;
			}
		}
		
		private void EndActivity()//wywolywana w funkcji Update()
		{
			if(currentActivityEndTime < clock.GetHour())
			{
				currentSeat.SetOccupation(false);
				StartActivity();
			}
		}
	\end{lstlisting}
	\caption{Funkcja wyliczająca szansę na rozwinięcie się choroby.}
	%\label{fig:kod:Activity}
\end{figure}

% % % % % % % % % % % % % % % % % % % % % % % % % % % % % % % % % % % 
% Pakiet minted wymaga odkomentowania w pliku config/settings.tex   %
% importu pakietu minted: \usepackage{minted}                       %
% i specjalnego kompilowania:                                       %
% pdflatex -shell-escape praca                                      %
% % % % % % % % % % % % % % % % % % % % % % % % % % % % % % % % % % % 

%\begin{minted}[linenos,breaklines,frame=lines]{c++}
%if (_nClusters < 1)
%   throw std::string ("unknown number of clusters");
%if (_nIterations < 1 and _epsilon < 0)
%   throw std::string ("You should set a maximal number of iteration or minimal difference -- epsilon.");
%if (_nIterations > 0 and _epsilon > 0)
%   throw std::string ("Both number of iterations and minimal epsilon set -- you should set either number of iterations or minimal epsilon.");
%\end{minted}
