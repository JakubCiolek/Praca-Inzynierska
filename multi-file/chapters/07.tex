\chapter{Podsumowanie i wnioski}

Praca nad projektem symulacyjnym \textit{Infektosym} przyniosła satysfakcjonujące wyniki, spełniające wszystkie postawione wymagania funkcjonalne. Pomimo osiągnięcia sukcesu w realizacji większości celów, pojawiły się pewne ograniczenia, szczególnie związane z liczbą populacji. Symulowanie dużej liczby osób w biurze, przekraczającej 100-150, okazało się niepraktyczne, co stanowi obszar do dalszych rozważań.

\begin{itemize}
	\item \textbf{Osiągnięte Cele i Wymagania} \\
	 W ramach pracy udało się z powodzeniem zrealizować założone cele oraz spełnić postawione wymagania funkcjonalne. Skonstruowany model symulacyjny pozwolił na wierne odwzorowanie procesu rozprzestrzeniania się chorób w środowisku biurowym.
	
	\item \textbf{Ograniczenia i Potencjalne Rozwinięcia} \\
	 Pojawiły się pewne ograniczenia związane głównie z liczbą symulowanych osób. W celu lepszej skalowalności należałoby rozważyć alternatywne scenariusze lub większe mapy. Dodatkowo, rozbudowa symulacji zachowań ludzkich oraz dodanie większej liczby scenariuszy to obszary, które mogą znacząco wzbogacić funkcjonalność aplikacji.
	
	\item \textbf{Ewentualne Kierunki Rozwoju} \\
	 Przyszłe prace nad projektem mogłyby skupić się na ulepszaniu symulacji zachowań ludzkich, wprowadzając bardziej złożone scenariusze i interakcje między agentami. Dodatkowo, warto rozważyć rozbudowę funkcjonalności o nowe scenariusze, co umożliwi bardziej wszechstronne testowanie i analizę zachowań w różnych warunkach.
\end{itemize}

W świetle uzyskanych wyników, projekt \textit{Infektosym} stanowi solidną podstawę do dalszych prac rozwojowych, mających na celu jeszcze bardziej zaawansowane modele symulacyjne i bogatsze doświadczenia użytkowników. 
