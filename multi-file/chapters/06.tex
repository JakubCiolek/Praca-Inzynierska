\chapter{Weryfikacja i walidacja}
\label{ch:06}
\begin{itemize}
\item sposób testowania w ramach pracy (np. odniesienie do modelu V)
\item organizacja eksperymentów
\item przypadki testowe zakres testowania (pełny/niepełny)
\item wykryte i usunięte błędy
\item opcjonalnie wyniki badań eksperymentalnych
\end{itemize}

W tym rozdziale przedstawione zostaną techniki sprawdzania i potwierdzania poprawności działania aplikacji. W procesie tworzenia oprogramowania istotne jest nie tylko zaplanowanie i implementacja funkcji, ale także dokładne sprawdzenie, czy rezultaty są zgodne z założeniami projektowymi. Weryfikacja koncentruje się na sprawdzeniu, czy projekt spełnia założenia funkcjonalne i techniczne, natomiast walidacja ocenia, czy to, co zostało zaimplementowane, odpowiada rzeczywistym potrzebom użytkowników. W dalszej części tego rozdziału przedstawione zostaną metody, narzędzia i procedury stosowane w procesie weryfikacji i walidacji.\\

\section{Sposoby testowania i organizacja eksperymentów}

W ramach przeprowadzonych testów aplikacji dokonano ręcznego dostosowania parametrów symulacji. Obejmowały one zakres od maksymalnych do minimalnych wartości oraz różne kombinacje pomiędzy nimi. Kluczowe parametry zostały testowane oddzielnie, a przykładowo długość symulacji poddana była analizie, aby ocenić wpływ zwiększenia liczby cykli dnia i nocy na funkcjonalność. Dodatkowo przetestowano różne wartości procentowe dla zarażonych i osób noszących maseczki. Eksperymentowano również z różnymi liczebnościami populacji, aby zrozumieć, jak wpływają one na symulację zachowań ludzkich.

%
% znalezione błedy 
% po pierwszym dniu maski przestają się pokazywać 
% agenci wypadaja z navmesh
% problem z rezerwowanieiem i zwalnianiem miejsc siedzących

