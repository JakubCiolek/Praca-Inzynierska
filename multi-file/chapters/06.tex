\chapter{Weryfikacja i walidacja}
\label{ch:06}
%\begin{itemize}
%\item sposób testowania w ramach pracy (np. odniesienie do modelu V)
%\item organizacja eksperymentów
%\item przypadki testowe zakres testowania (pełny/niepełny)
%\item wykryte i usunięte błędy
%\item opcjonalnie wyniki badań eksperymentalnych
%\end{itemize}

W tym rozdziale przedstawione zostaną techniki sprawdzania i potwierdzania poprawności działania aplikacji. W procesie tworzenia oprogramowania istotne jest nie tylko zaplanowanie i implementacja funkcji, ale także dokładne sprawdzenie, czy rezultaty są zgodne z założeniami projektowymi. Weryfikacja koncentruje się na sprawdzeniu, czy projekt spełnia założenia funkcjonalne i techniczne, natomiast walidacja ocenia, czy to, co zostało zaimplementowane, odpowiada rzeczywistym potrzebom użytkowników. W dalszej części tego rozdziału przedstawione zostaną metody, narzędzia i procedury stosowane w procesie weryfikacji i walidacji.\\

\section{Sposoby testowania i organizacja eksperymentów}

W ramach przeprowadzonych testów aplikacji dokonano ręcznego dostosowania parametrów symulacji. Obejmowały one zakres od maksymalnych do minimalnych wartości oraz różne kombinacje pomiędzy nimi. Kluczowe parametry zostały testowane oddzielnie, a przykładowo długość symulacji poddana była analizie, aby ocenić wpływ zwiększenia liczby cykli dnia i nocy na funkcjonalność. Dodatkowo przetestowano różne wartości procentowe dla zarażonych i osób noszących maseczki. Eksperymentowano również z różnymi liczebnościami populacji, aby zrozumieć, jak wpływają one na symulację zachowań ludzkich.


\section{Przypadki testowe}

Tabela \ref{tab:parameters} przedstawia różnorodne przypadki testowe dla parametrów symulacji w aplikacji. Testy zostały zaplanowane w celu sprawdzenia zachowania symulacji w ekstremalnych warunkach oraz w warunkach typowych. Poniżej znajduje się krótki opis każdego testu:

\begin{itemize}
	\item \textbf{Test 1 (Maksymalne wartości):} Ten test uwzględnia maksymalne wartości dla każdego z parametrów symulacji. Jest to scenariusz ekstremalny, mający na celu zidentyfikowanie, czy aplikacja poprawnie obsługuje ekstremalne warunki, takie jak bardzo zaraźliwe środowisko, długi czas inkubacji czy duża liczba populacji.

	\item \textbf{Test 2 (Minimalne wartości):} W tym teście wszystkie parametry przyjmują minimalne wartości. Celem jest sprawdzenie, czy aplikacja jest odporna na sytuacje, w których czynniki zakaźne są minimalne, a populacja jest ograniczona.
	
	\item \textbf{Test3 (Wartości domyślne):} Test ten obejmuje wartości domyślne dla wszystkich parametrów symulacji. Stanowi punkt odniesienia, w którym sprawdzane jest standardowe zachowanie symulacji.
	
	\item \textbf{Testy 4 - 8 (Losowe wartości):}
	 Pozostałe testy obejmują losowo wybrane wartości parametrów symulacji. Te przypadki mają na celu ocenę ogólnej elastyczności i skuteczności aplikacji w zróżnicowanych warunkach. Wartości te zostały dobrane w celu odtworzenia różnorodnych scenariuszy, które mogą wystąpić w rzeczywistych warunkach.
\end{itemize}
	
\begin{table}[h!]
	\caption{Parametry symulacji użyte w poszczególnych testach.}
	\centering
	\begin{tabular}{|c|c|c|c|c|c|c|c|c|}
		\hline
		\textbf{Parametry Symulacji} & \textbf{Test1} & \textbf{Test2}  & \textbf{Test3}  & \textbf{Test4}  & \textbf{Test5} & \textbf{Test6} & \textbf{Test7} & \textbf{Test8}  \\
		\hline
		Dystans Zarażenia & 3 & 1 & 2 & 4 & 5 & 1 & 3 & 2\\
		\hline
		Czas do Zarażenia & 60 &  1 & 2 & 30 & 10 & 15 & 40 & 5\\
		\hline
		Zaraźliwość Patogenu & 100  & 0 & 50 & 75 & 20 & 40 & 60 & 10\\
		\hline
		Średni Okres Inkubacji & 3 & 1 & 2 & 1 & 3 & 1 & 2 & 3\\
		\hline
		Procent Noszący Maseczki & 100 & 0 & 0 & 50 & 80 & 20 & 70 & 40\\
		\hline
		Skuteczność Maseczek & 70 & 0 & 30 & 50 & 90 & 10 & 60 & 30\\
		\hline
		Liczebność Populacji & 150 & 10& 30 & 100 & 50 & 20 & 120 & 80\\
		\hline
		Procent Zarażonych & 50 & 1 & 5 & 30 & 10 & 2 & 20 & 8\\
		\hline
		Odporność Populacji & 90 & 0 & 50 & 70 & 30 & 20 & 80 & 40\\
		\hline
		Długość Symulacji & 10 & 1 & 2 & 5 & 8 & 1 & 7 & 3\\
		\hline
		Średni czas wykrycia & 3 & 1 & 2 & 1 & 3 & 1 & 2 & 3\\
		\hline
	\end{tabular}
	\label{tab:parameters}
\end{table}


\section{Wykryte i usunięte błędy}

Zidentyfikowane i skorygowane błędy obejmują:
\begin{itemize}
	\item Po pierwszym cyklu dnia i nocy, stwierdzono kompletną utratę maseczek. Problem wynikał z braku wywołania funkcji \textit{ShowMask()} w metodzie inicjującej symulację po zakończeniu nocy.
	\item Wystąpiły problemy z rezerwacją i zwalnianiem miejsc siedzących, gdzie agenci często próbowali zająć te same miejsca. Problem ten został poprzez zmianę logiki stojącej za mechanizmem. Zamiast przechowywania zajętego miejsca przez człowieka, teraz to siedzenie przechowuje informacje, o tym który człowiek na nim siedzi.
	\item Po przeprowadzeniu rozmowy, agenci przestawali mieć jakiekolwiek interakcje z innymi. Spowodowane to było omyłkowym wywołaniem metody zmniejszającej dystans zarażenia w funkcji obsługującej rozmowę.
\end{itemize}

%
% znalezione błedy 
% po pierwszym dniu maski przestają się pokazywać 
% agenci wypadaja z navmesh
% problem z rezerwowanieiem i zwalnianiem miejsc siedzących

