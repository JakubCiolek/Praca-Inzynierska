\subsubsection*{Tytuł pracy} 
\Title

\subsubsection*{Streszczenie}  
Na podstawie uznanych modeli rozprzestrzeniania się zarażeń oraz analizy istniejącego oprogramowania badającego epidemie, opracowano aplikację symulacyjną, której głównym celem jest ukazanie w czasie rzeczywistym sposobu rozprzestrzeniania się chorób w środowisku biurowym. Aplikacja łączy zmodyfikowany matematyczny model symulacji z podejściem agentowym, co zapewnia bardziej precyzyjne odwzorowanie rzeczywistości. Szczególną uwagę skupiono na dostarczeniu prostego i zrozumiałego interfejsu wizualnego, umożliwiającego zrozumienie mechanizmów stojących za  zjawiskami epidemiologicznymi nawet osobom nieposiadającym specjalistycznej wiedzy medycznej.

\subsubsection*{Słowa kluczone}
Modelowanie epidemii,
Aplikacja edukacyjna,
Bezpieczeństwo zdrowotne,
Interaktywne narzędzie edukacyjne,
Matematyczne modele epidemiologiczne

\subsubsection*{Thesis title} 
\begin{otherlanguage}{british}
\TitleAlt
\end{otherlanguage}

\subsubsection*{Abstract} 
\begin{otherlanguage}{british}
Based on recognized models of infection spread and an analysis of existing epidemic research software, a simulation application has been developed. Its primary goal is to showcase, in real-time, the way diseases spread in a office environment. The application combines a modified mathematical simulation model with an agent-based approach, providing a more precise representation of reality. Special attention has been given to delivering a simple and understandable visual interface, allowing individuals without specialized medical knowledge to comprehend the mechanisms behind epidemiological phenomena.
\end{otherlanguage}
\subsubsection*{Key words}  
\begin{otherlanguage}{british}
	Epidemic modeling,
	Educational application,
	Health safety,
	Interactive educational tool,
	Mathematical epidemiological models
\end{otherlanguage}

