\chapter{Wstęp}
\label{ch:wstep}

W świecie, który dopiero co doświadczył globalnej pandemii COVID-19, zauważamy potrzebę skutecznych narzędzi zarówno do przewidywania rozprzestrzeniania się infekcji, jak i podnoszenia świadomości społeczeństwa na temat konieczności przestrzegania restrykcji i ochrony zdrowia. Pandemia wywołała potrzebę innowacyjnych rozwiązań, w obszarze przewidywania i zrozumienia dynamiki rozprzestrzeniania się wirusa. W kontekście informatyki, praca skupia się na wykorzystaniu komputerów do opracowania aplikacji, która nie tylko pozwala na modelowanie dynamicznych scenariuszy rozprzestrzeniania się wirusa, ale także stawia na edukację społeczną w zakresie efektywnych praktyk prewencyjnych\cite{bib:simulationforeducation}.

Dziedzina informatyki w znacznym stopniu przyczynia się do rozwiązania problemów związanych z pandemią, dowodem na to jest niezliczona ilość artykułów naukowych związanych z tym tematem opublikowanych w czasie pandemii\cite{bib:covid19}\cite{bib:covid191}\cite{bib:covid192}. Mając do dyspozycji technologię, jesteśmy w stanie opracować zaawansowane algorytmy symulacyjne, które umożliwiają modelowanie złożonych interakcji społecznych i ruchu ludzi w różnych środowiskach. Komputery stają się potężnym narzędziem do analizy danych, identyfikowania wzorców i prognozowania potencjalnych scenariuszy rozprzestrzeniania się infekcji.

Symulacje komputerowe pozwalają nam przewidywać, jak różne warunki środowiskowe i społeczne wpływają na tempo i zasięg rozprzestrzeniania się wirusa. Dodatkowo, umożliwiają szybkie testowanie różnych scenariuszy i strategii reakcji na sytuacje kryzysowe, co przyczynia się do skuteczniejszego przygotowania się do potencjalnych zagrożeń zdrowotnych\cite{bib:Brockmann2017Global}.

Niniejsza praca w obszarze informatyki nie tylko skupia się na technicznej strukturze aplikacji, ale również na zastosowaniu narzędzi informatycznych w celu zwiększenia świadomości społecznej. Komputery służą jako platforma, na której możemy nie tylko symulować scenariusze, ale także efektywnie komunikować się z użytkownikami spoza środowiska medycznego, edukując ich na temat istoty zachowania się w sposób, który zmniejsza ryzyko zakażenia.

Rozdział drugi skupiony jest na przeglądzie istniejących modeli symulacji zarażeń, opartym na dogłębnej analizie dostępnej literatury.
Ma na celu zidentyfikować różne podejścia i metody, które zostały wykorzystane w modelowaniu rozprzestrzeniania się infekcji. 

Rozdział trzeci posłuży do przedstawienia wymagań projektowych i narzędzi, które posłużą do ich realizacji. Analiza potrzeb funkcjonalnych i technicznych pozwoli na wybór odpowiednich technologii i narzędzi programistycznych

W kolejnych dwóch rozdziałach przedstawiono odpowiednio specyfikacja zewnętrzna i wewnętrzna aplikacji. Zdefiniowany zostanie interfejs użytkownika, funkcjonalności dostępne dla użytkowników końcowych oraz scenariusze użycia. Następnie opis architektury, struktury kodu i wszystkich kluczowych elementów wewnętrznych. 

Szósty rozdział poświęcony został procesom weryfikacji i walidacji stworzonej aplikacji. Opisuje wykorzystane scenariusze testowe.

Ostatni rozdział to podsumowanie całej pracy, uwzględniające wnioski powstałe z realizacji projektu oraz ewentualne kierunki dalszych rozwojów.
