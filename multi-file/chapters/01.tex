\chapter*{Wstęp}
\label{Wstęp}

W świecie, który dopiero co doświadczył globalnej pandemii COVID-19, zauważamy potrzebę skutecznych narzędzi zarówno do przewidywania rozprzestrzeniania się infekcji, jak i podnoszenia świadomości społeczeństwa na temat konieczności przestrzegania restrykcji i ochrony zdrowia. Pandemia stworzyła potrzebę innowacyjnych rozwiązań, w obszarze przewidywania i zrozumienia dynamiki rozprzestrzeniania się wirusa. W kontekście informatyki, praca skupia się na wykorzystaniu komputerów do opracowania aplikacji, która nie tylko pozwala na modelowanie dynamicznych scenariuszy rozprzestrzeniania się wirusa, ale także stawia na edukację społeczną w zakresie efektywnych praktyk prewencyjnych \cite{bib:simulationforeducation}.

Dziedzina informatyki w znacznym stopniu przyczynia się do rozwiązania problemów związanych z pandemią. Dowodem na to jest niezliczona ilość artykułów naukowych związanych z tym tematem opublikowanych w czasie pandemii \cite{bib:covid19}\cite{bib:covid191}\cite{bib:covid192}. Mając do dyspozycji technologię, jesteśmy w stanie opracować zaawansowane algorytmy symulacyjne, które umożliwiają modelowanie złożonych interakcji społecznych i ruchu ludzi w różnych środowiskach. Komputery stają się potężnym narzędziem do analizy danych, identyfikowania wzorców i prognozowania potencjalnych scenariuszy rozprzestrzeniania się infekcji.

Symulacje komputerowe pozwalają nam przewidywać, jak różne warunki środowiskowe i społeczne wpływają na tempo i zasięg rozprzestrzeniania się wirusa. Dodatkowo, umożliwiają szybkie testowanie różnych scenariuszy i strategii reakcji na sytuacje kryzysowe, co stwarza sposobność do skuteczniejszego przygotowania się do potencjalnych zagrożeń zdrowotnych \cite{bib:Brockmann2017Global}.

Celem niniejszej pracy jest stworzenie aplikacji, dedykowanej symulacji rozprzestrzeniania się epidemii, ze szczególnym uwzględnieniem środowiska biurowego. Głównym założeniem projektu jest dostarczenie narzędzia umożliwiającego szeroką parametryzację symulacji, co pozwoli na elastyczne dostosowanie scenariuszy do różnych warunków i kontekstów. Aplikacja ma stanowić realistyczny model symulacyjny, uwzględniający codzienne zachowania ludzkie w biurze, a także umożliwiać interaktywną wizualizację procesu rozprzestrzeniania się patogenów w czasie rzeczywistym.

Pracę podzielono na siedem rozdziałów. Na początku pracy dokonano analizy istniejących modeli symulacji zarażeń. Następnie sprecyzowano wymagania projektowe i narzędzia, które posłużyły do realizacji projektu. W dalszej części opisano specyfikacje wewnętrzną i zewnętrzną programu oraz scenariusze dotyczące walidacji i testowania. Na końcu  przedstawiono wnioski i dalsze możliwości rozwoju aplikacji.
